\documentclass{article}
\usepackage{graphicx} % Required for inserting images
\usepackage[a4paper, margin=1in]{geometry}
\usepackage{amssymb}
\usepackage{amsmath} 
%%
\usepackage[margin=1in]{geometry}

%Esto es para las lineas de "Paradoja de la Urna - Probabilidad I"
\usepackage{fancyhdr}
\pagestyle{fancy}
\fancyhead[L]{Notes}
\fancyhead[R]{Mates Actuariales}
\fancyfoot[C]{\thepage}
%%

\title{Matemáticas Actuariales}
\author{Héctor Astudillo}
\date{February 2025}

\begin{document}

\maketitle

\section*{Flujo contingente}
El flujo contingente se define de la siguiente manera
\((\$,\mathbb{P}(\$),v^t )\)
\\
Lo que voy a pagar, probabilidad de pagar, cuanto vale hoy


\subsection*{Notación - Cantidad de Existencia ó Tiempo Futuro de Vida}
Notación:
\begin{itemize}
    \item \((x)\) - Representa a una persona de edad \(x\)
    \item \(t\) - Indica tiempo
    \item \(T_x\) Tiempo de vida futuro de una persona de edad \(x\)
\end{itemize}

Entonces, \(T_{20} = 55\) se lee como \textbf{"El tiempo de vida futuro de una persona de edad 20"}

Tal que, en otras palabras, estamos indicando una posible muerte a la edad de 20+55= 75 años.
\\\\
La idea principal es calcular probabilidades, entonces nos preguntamos ¿Cuál es la probabilidad de que una persona con edad 20 años viva 55 años más?
\\\\
De nuestras clases de probabilidad (e intuitivamente) como mencionamos, daremos uso del concepto de tiempo, para Newton el tiempo lo definía como continuo, entonces es por ello que estamos hablando del cálculo de \textbf{probabilidades con variables continuas}.
\\\\
Recordando que, para variables continuas, NO medimos la probabilidad sobre un punto, en un punto tendría probabilidad 0, \textbf{para este tipo de variables usamos intervalos}.
\\\\
\textbf{El concepto clave son INTERVALOS}

\section*{Funciones de Distribución}
\subsubsection*{F - Función de fallecimiento}
Notación:
\begin{itemize}
    \item \(F_T\) - Mide muerte, \textbf{fallecimiento} 
    \item \(F_{T_x}(t)\) - Función de distribución tal que \textbf{calcula la probabilidad acumulada en un intervalo}.
\end{itemize}
Entonces si yo escribo:

\(F_{T_x}(t)=\mathbb{P}(T_x \leq t ),\;\forall t\)

Nos referimos a:

\textit{"La probabilidad que el tiempo futuro de vida de una perrsona con edad \((x)\) \textbf{sea menor o igual a t años, edad x+t}".}
\\

Tambien se puede leer como:

\textit{"La probabilidad de que una persona de edad x fallezca a la edad de x+t o t años"}
\\\\
Entonces recuerda que la clave eran los intervalos, si yo escribo \(\mathbb{P}(T_{36} \leq 4)\)

Me estoy refiriendo al concepto de que, una persona de edad 36, fallezca dentro de 4 años o, con edad 36+4=40 años.
\\
Si el intervalo va hacia la derecha, nos referimos a tiempo de fallecer, al concepto de fallecer concretamente.

\textbf{Notación:}
\begin{itemize}
    \item \(F_x(t) = F_{T_x}(t)\)
    \item \(t\) -Indica el final de mi línea de tiempo que será dividido en subintervalos tales que serán calculados.
    \item \({}_tq_x\) - En notación actuarial
\end{itemize}
\subsubsection*{S - Función de Sobrevivvencia}

De la misma manera que \(F\), \(S\) es mi función de distributividad que mide en relación a probabilidades acumuladas pues mi variable sobre la cual estoy midiendo es el tiempo, una variable aleatoria continua.
\begin{itemize}
    \item \(S_x(t) = S_{T_x}(t)\)
    \item \({}_tp_x\) - En notación actuarial
    \item \(S_x(t) = S_{T_x}(t) = 1-F_x(t)\)
    \item \(S_x(t) = \mathbb{P}(t<T_x)\) \textbf{***}
    
\end{itemize}

Entonces, \(S_x(0)\) nos referimos a la probabilidad de supervivencia de una persona con edad x, tal que este viva en este momento, en tiempo 0 a partir de ahorita.
\begin{itemize}
    \item \(S_x(0) = 1\) - Pues esta viva la persona en el momento.
    \item Graficamente sobre una recta que me represente el tiempo, podemos pensar que en el extremo izquierdo tengo que la función de supervivencia \(S_x(t)\), y en el extremo derecho la función de fallecimiento \(F_x(t)\) tal que todo ese intervalo en un sentido equivale a 1, de inicio a fin, tenemos asegurada la muerte.
\end{itemize}
Analiza lo siguiente:
\[\lim_{t\rightarrow\infty} S_x(t)=0\]

Nos esta diciendo que cuando t, la edad en la que me pregunto si podría sobrevivir tiende o corre a infinito, tiempo de sobrevivencia infinito, es 0, nadie sobrevive t infinitos años, si lo decimos en algún sentido.

\subsection*{Función de Fallecimiento Distributiva}
Como vimos \(F_x(t) = \mathbb{P}(T_x \leq t)\) - Como bien \textbf{lo definimos} anteriormente, recuerda que el valor de fallecimiento va a la derecha, por ende usamos el menor igual.

\subsection*{Porb. Condicional - Interpretar la Probabilidad Acumulativa}

Dado \(F_x(t) = \mathbb{P}(T_0 \leq x+t)\) - \textbf{Nota que estamos hablando de un recien nacido}
\\\\
Probabilidad de que una persona recien nacida fallezca dentro de \(x+t\) años o a la edad de \(x+t+0\) años... 
\\
\textbf{Observación:}

\(F_x(x+t) = \mathbb{P}(T_0 \leq x+t)\) Esto en otro sentido, significa que queremos calcular \(T_o \leq x+t\;|\;0<T_x\)
\\\\
\textbf{Clave: Recuerda lo que mencionamos de imaginar nuestra línea de tiempo en donde a la derecha se coloca el fallecimiento y a la izquierda años vividos, sobrevivencia}
\\\\
Entonces si generalizamos todo (observa que en este caso el recien nacido ya vivió x años)... 
\\
\(F_x(t) = \mathbb{P}(T_0 \leq x+t) = \mathbb{P}(T_0 \leq x+t) | x<T_o)\)

Por definición:

\[=\frac{\mathbb{P}(T_0 \leq x+t) \;\cap\; x<T_o)}{\mathbb{P}(x<T_o)} = \frac{\mathbb{P}(x< T_0 \leq x+t)}{\mathbb{P}(x<T_o)}\]

Ahora recuerda la gráfica, lo que queremos en este caso, dado que ya vivió x años este recien nacido...\textbf{queremos restarle a nuestro cálculo de fallecimiento, sus años ya vivivos}
\\\\
\textbf{Ten cuidado con el signo negativo, cambia desigualdades!} Pasamos de tener \(\mathbb{P}(x<T_0...)\) a \(...-\mathbb{P}(x>T_0)\)
\\\\
\textbf{Ojo que nuestro denominador hace alusión al tiempo ya sobrevivido por \(T_o\), osea que es la función de distribución de sobrevivencia.}

\[= \frac{\mathbb{P}(T_0 \leq x+t) - \mathbb{P}(x>T_0)}{S_{T_x}(x)}\]

En nuestro numerador, ya tenemos solo funciones respecto a la función de fallecimiento.

\[=\frac{F_0(x+t)-F_0(x)}{S_0(x)}\]

Tal que lo podemos escribir en términos de S, la función de sobrevivencia añadiendo un 0 (+1-1)
\[=\frac{F_0(x+t)-F_0(x)+1-1}{S_0(x)}= 
\frac{F_0(x+t)-1+1-F_0(x)}{S_0(x)} =
\frac{-(1-F_0(x+t))+1-F_0(x)}{S_0(x)}\]

\[=\frac{-(S_{T_0}(x+t))+S_{T_0}(x)}{S_{T_0}(x)}=
\frac{S_{T_0}(x)-(S_{T_0}(x+t))}{S_{T_0}(x)}=
\frac{S_{T_0}(x)}{S_{T_0}(x} - \frac{S_{T_0}(x+t)}{S_{T_0}(x)} = 
1 - S_{T_0}(t) = 1-{}_t P_0\]

\[\therefore F_x(t) = 1- {}_tP_0\]
\subsection*{\(S_x\) en términos de \(S_0\)}
\textbf{De lo anterior se observa que \[S_{T_x}(t)=\frac{S_{T_0}(x+t)}{S_{T_0}(x)} = {}_tP_x\] Lo importante a observar es que \(T_x\) está en terminos de \(T_0\)}
\\
En cierto sentido, una persona de edad x la estamos llevando a su edad recien nacida (0) y estamos calculando la probabilidad de su sobrevivencia a edad \(x\) años más \(t\) años. Es más fácil de interpretar dentro de una gráfica. \textbf{\textit{O bien interpreta esa resta de intervalo como una división.}} Esta es la clave para este caso.
\\
\textbf{Entonces \[S_x(t+u) =\frac{S_0(x+t+u)}{S_0(x)}\]}
\\
Del cual se obtiene que \[= \frac{S_0(x+t+u)}{S_0(x)} \cdot 1 = \frac{S_0(x+t+u)}{S_0(x)} \cdot \frac{S_0(x+t)}{S_0(x+t)}\]
\\
Que por propiedades conmutativas \[=\frac{S_0(x+t+u)}{S_0(x+t)}\cdot\frac{S_0(x+t)}{S_0(x)} = S_{t+x}(u)\cdot S_x(t)\]

\[\therefore {}_{t+u}P_x = {}_uP_{x+t}\cdot {}_tP_{x}\]

\suubsection*{Prob. DIFERIDA}
"\textbf{La probabilidad de que una persona con edad \(x\) sobreviva \(t\) años y muerea dentro del intervalo de edad x+t y x+t+u}".
\\
Esto en términos de probabilidad como vimos es:
\[\mathbb{P}(t<T_x\leq t+u)\]

\textbf{¿Por qué \(t+u\)?}

Recuerda que dentro de la probabilidad \(\mathbb{P}\) t y u estan dados en EDAD, por ende \textbf{\textit{t se mantiene como la edad de sobrevivencia y t+u como la edad de fallecimiento. }}
\\

\textbf{NO CONFUNDIRSE AL INTERPRETARLO DENTRO DE UNA LÍNEA DE TIEMPO}

Al inicio está \((x)\), \(T_x\) \textbf{NO está al inicio de nuestra linea pues como tal es el TIEMPO DE VIDA FUTURO, este concepto se mueve dentro de estos intervalos, por eso lo centramos en las probabilidades}, seguido de \(t\) que es su intervalo de sobreviencia y seguido de su intervalo de fallecimiento t+u

Esto claramente en términos de edad para \(\mathbb{P}\)
\\

Recuerda como extra que \textbf{\textit{el valor de la función cuando convertimos una de sobrevivencia a fallecimiento o viceversa este se mantiene, no cambia, no se resta ni nada, el valor es el mismo al cambiar de función.}}
\[S_x(1) = 1-F_x(1) \]
\\\\
\textbf{Calcularemos entonces, la probabilidad de que una persona de edad x sobreviva t años y muera dentro del intervalo x+t y x+t+u vista desde la función de sobrevivencia y la función de fallecimiento}

\begin{itemize}
    \item Hablamos de probabilidades condicionales, muera dado que sobrevivió
    \item \(\mathbb{P}(t<T_x\leq t+u) = \mathbb{P}(T_x < t+u) - \mathbb{P}(T_x < t)\)
        \item Donde \(\mathbb{P}(T_x < t+u)\) es el intervalo completo, desde x hasta \(t+u\) (prob. de que fallezca hasta edad x+t+u) y \(\mathbb{P}(T_x < t)\) es la prob. de fallecer a edad x+t PERO esto se resta porque esa prob. ya pasó. CONCEPTO DE DIFERIDO.
        \item Por lo anterior, que nos basamos en probabilidades de fallecer obtenemos: \(F_x(t+u) - F_x(t)\)
        \item Recuerda que: \(F_x(t) = 1- S_x(t)\)
        
        \(\Rightarrow F_x(t+u) - F_x(t) = 1-S_x(t+u) - (1-S_x(t))= -S_x(t+u)+S_x(t) = S_x(t) - S_x(t+u)\)
    \item Notación Act. \(\rightarrow {}_{t|u}P_x\)
\end{itemize}

\subsection*{Propiedades de la función de sobrevivencia (\(S_x\))}
En palabras:
\begin{itemize}
    \item La probabilidad de que una persona de edad x esté con vida en edad x+0 (ahorita) es 1.
    \item La función es decrececiente, sus valores en y solo pueden mantenerse y/o decrecer.
    \item Una persona no puede vivir años infinitos, ser inmortal
\end{itemize}
En notación matemática:
\begin{itemize}
    \item \({}_0P_{x} = 1\)
    \item Si \(v<u\Rightarrow f(v)\leq f(u)\)
    \item \(\lim_{t\rightarrow\infty} S_x(t) = 0\)
\end{itemize}

\section*{FUERZA de Mortalidad}
Se define como la \textbf{tasa} instantanea de \textbf{fallecimiento} al momento t para una persona de edad x, esta definición es para MASPI 1.
\\
Observa los conceptos de tasa y fallecimiento, por el cual estamos calculando la prob. de muerte de la persona en el intervalo \((x,x+t)\).
\\\\
Para este tema necesitamos recordar:
\begin{itemize}
    \item Probabilidad Condicional
    \item Si \(f(u)=ln(u) \Rightarrow f'(u)=\frac{u'}{u}\)
\end{itemize}
Notación:
\textbf{La fuerza de mortailidad para una persona de edad t}
\begin{itemize}
    \item \(M_t\)
\end{itemize}
Entonces, la fuerza de mortalidad de una persona de edad x+t
\[M_{x+t}{lim_{}\]
PENDIENTE POR TERMINAR

\section*{Tabla de vida}
Es una herramienta demográfica que nos ayuda a describir la supervivencia y mortalidad de una población. Nos ayudan a modelar y calcular las probabilidades de vida o muerte.

\subsection*{l - living}
Recuerda que:
\begin{itemize}
    \item Media muestral es tu media real, digamos la obtienes de la muetra real que estas estudiando
    \item Media teórica es la que se calcula, recuerdalo como que en la parte teorica se realizan los cálculos.
    \item Ley de grandes números
\end{itemize}
Se dice que para un número suficientemente grande de personas expuestas, el número de muertos no es completamente aleatorio, si no que siguen comportamientos promedios.

Predecimos el promedio mas NO a quienes.
\\\\
Notación:
\begin{itemize}
    \item \textbf{Analiza que con los subíndices que acompañen el simbolo se refieren a edades.}
    \item \(L_0\) = \textbf{NÚMERO INICIAL de personas en el grupo de observación, de edad 0} = "Radix"
    \item \(-hip\) = El grupo inicial de observados = \(L_0\)
    \item \(L_x\) = Número de personas vivas a edad x (EN REALIDAD ES ENTRE EDADES X y X+1 asi como la de \(d_x\))
    \item \(L_w\) - Definimos a \(w\) como la maxima edad teórica probable que vive una persona (en particular de mi grupo de observados)
    \item \(L_{w+1} = 0\)
\end{itemize}
Observemos que:
\begin{itemize}
    \item \(L\) es NO creciente, digamos que se PARECE, mas NO es la función de sobrevivencia, tal que \(L_x \geq L_{x+1},\;\forall x\)
\end{itemize}

\subsection*{d - deathing}
Notación:
\begin{itemize}
    \item \(d_x\) = \textbf{Número de salidas (muertes) entre edades x y x+1}. El tiempo de muerte se mide en intervalos, tal que puede suceder en cualquier punto del intervalo, teniendo cuidado de no sacar la probabilidad en ese punto específico pues obtendremos 0, recuerda, no decimos quien o inclusive cuando morirá una persona, predecimos el promedio y dentro de que intervalos es más probable.
\end{itemize}

\subsection*{Formulas}
\begin{itemize}
    \item El número de personas dentro de mi grupo de observados \textbf{a EDAD x+1} es igual al \textbf{número de personas vivas}, NO INICIALES, estamos usando x, osea \textbf{de edad x} MENOS el \textbf{número de salidas (muertes) de mi grupo entre edades x y x+1}. \[l_{x+1} = l_x - d_x\]
    
    \item El número de personas muertas \textbf{con edad 0} es igual a el numero de personas vivas de edad 0 MENOS el número de personas vivas en tiempo t = 1 ó edad 1. \[d_0 = l_0 - l_1\]

    \item Entonces, genralizando lo anterior tenemos que: \[d_x = l_x - l_{x+1}\] y que \[{}_td_x = l_x - l_{x+t}\]

    \item RECORDEMOS: Prob. Clásica \(\mathbb{P}(\cdot) = \frac{\text{Casos Favorables}}{\text{Casos Totales}}\). \[{}_1P_x = \frac{L_{x+1}}{L_x}\]
    \item En palabras, la prob. de que una persona de edad x sobreviva a la edad x+1 es igual a el numero de personas de mi grupo de observación vivas a edad x+1, (entre edades x y x+1) sobre el numero de personas vivas a edad x, (entre edad 0 y x)

    \item Entonces la probabilidad de que una persona de edad x sobreviva a edad x+t es \({}_tP_x\), que \textbf{en términos de funciones biométricas} sería \(l_{x+t}/l_x\)

    \item PROB DE FALLECIMIENTO \[q_x = 1-p_x\]
    \item En general, esto ya lo sabiamos pero desarrollandolo aún más con lo que acabamos de ver tenemos: \[q_x = 1- p_x  \Rightarrow q_x = 1 - \frac{l_{x+1}}{l_x} = \frac{l_x}{l_x} - \frac{l_{x+1}}{l_x}\]

    \item Generalizando entonces lo anterior: \[{}_tq_x = \frac{{}_td_x}{l_x}\]
\end{itemize}

Puede que suene raro el hecho de que estamos calculando la probabilidad de que UNA persona de edad x sobreviva a y años con base a cantidades que se refieren a un numero de personas (l ó d), pero vemos que hace sentido si por ejemplo, fuera el caso de que calcular la probabilidad de que en un grupo de 20 personas 10 de ellos saquen 10... sabiendo que al menos la mitad reprueba, entonces la probabilidad es 1/2, esa probabilidad en si es para una persona aleatoria, cada persona del grupo de estudio tiene la misma probabilidad.
\\\\
Entonces tenemos lo siguiente:
\begin{itemize}
    \item Numero de personas vivas en mi grupo de personas a edad x \[l_x\]
    \item Número de personas muertas en mi grupo de observados entre edad x y x+1 \[d_x = l_x - l_{x+1}\]
    \item Numero de personas muertas en mi grupo de edad x+t o bien entre los intervalos x y x+t. Recuerda que este numero de muertes en tiempo x se calcula en terminos de número de vivos, en edad x menos un intervalo de edad hacia adelante de x, digamos x+1.\[{}_td_x = l_x -l_{x+t}\]
    \item La probabilidad de que una persona de edad x sobreviva a a la edad x+1 es igual al calculo de la probabilidad clásica, \[\frac{d_x}{l_x}\]
\end{itemize}

\subsection*{Identificación de notación correcta de probabilidades}
Antes que nada observa como identificar correctamente los casos favorables en ambas probabilidades, de fallecimiento y sobrevivencia...
\[q_x = {}_1d_x\] Observa que en su caso favorable, digamos que la notación de el numero de personas muertas se mantiene igual, identifco.

\[p_x = l_{x+1}\] Observa que en su caso favorable, la notación si cambia respecto al numero de observados vivos
\\\\
Para los casos totales siempre hablamos del numero de personas \textbf{vivas} en mi grupo con base a su edad, osea el numero de personas vivas de edad... x como es en estos ejemplos.

\subsection*{Probabilidad diferida}
Dado \[{}_{t|1}q_x\]
Esto se lee como: La probabilidad de que una persona de edad x fallezca entre las edades x+t y x+t+1 dado que ya sobrevivió a la edad x+t
\\
Entonces podemos reinterpretarlo con base al enunciado anterior de como se lee como:
\[{}_{t|1}q_x = {}_tp_x \cdot {}_1q_{x+t}\]
Recuerda que en cierto sentido, si queremos sumar intervalos de nuestra recta si lo estamos viendo asi, esa suma es más bien una multiplicación.
\\\\
Veamos que \[{}_{t|1}q_x = {}_tp_x \cdot {}_1q_{x+t} = \frac{l_{x+t}}{l_x} \cdot \frac{d_{x+t}}{l_{x+t}}=\frac{d_{x+t}}{l_x}\] Pues se cancela \(l_{x+t}\)
\\
Pero además veamos que:
\[\frac{d_{x+t}}{l_x} = \frac{l_{x+t}-l_{x+t+1}}{l_x} = \frac{l_{x+t}}{l_x}- \frac{l_{x+t+1}}{l_x} = {}_tP_x - {}_{t+1}P_x = S_x(t)-S_x(t+1)\]

\subsubsection*{Equivalencias de Probabilidad Diferida}
\textbf{Por ende no solo debemos de ver la probabilidad diferida como una multiplicación, esto es así en el caso particuar cuando trabajamos con \(P\) y \(q\), probabilidades. Para el caso de trabajar con conceptos de la tabla de vida es aquí donde podemos expresarlo como una resta (claro que la deducción sale a parti de la equivalencia con probabilidades, una multiplicación).}

\subsection*{Propiedad recursiva de \({}_tp_x\)}
Esto se interpreta de mejor manera dentro de una recta, una linea de tiempo.
Empezamos desde la edad x, tal que nuestra recta, sus valores van avanzando de 1 en 1, refiriendonos a años claro, entonces tendríamos los valores x, x+1, x+2, etc.
\\

Entonces de edad x a x+1, la P probabilidad de sobrevivir es \({}_1p_x\)
\\

Para el intervalo de edad x+1 a x+2, estemos avanzando una unidad, entonces es \({}_1p_{x+1}\), pero si yo lo que quiero es calcular la prob. de que sobreviva a edad x+2 osea \({}_2p_x\) no me basta simplemente ese intervalo, tengo que añadirle el otro intervalo que ya sobrevivió! Osea de x a x+1, este es el anterior punto, entonces \textbf{recuerda que si queriamos añadir o "sumar" esos intervalos, haciamos una multiplicación}. Entonces obtenemos lo siguiente: \[{}_2p_{x} = {}_1p_x \cdot {}_1p_{x+1} = {}_0p_x \cdot {}_2p_x\]

Entonces \[{}_3p_x = {}_2p_x \cdot {}_1p_{x+2}\]

Que de manera general tenemos que \[{}_tp_x = {}_{t-1}p_x \cdot {}_1p_{x+(t-1)} = \pi_{t=0}^{t-1} p_{x+t}\]
Que de manera coloquial decimos que estamos uniendo intervalos.

\subsubsection*{Interpretación de "restar" un intervalo}

Recordemos que:
\[{}_{t|1}q_x = {}_{t}P_x \cdot {}_{1}q_{x+t}\]
Que con base a los conceptos de tabla de vida \(l_x, d_x\) obtenemos que:
\[ {}_{t}P_x \cdot {}_{1}q_{x+t} = \frac{l_{x+t}}{l_x} \cdot \frac{d_{x+t}}{l_{x+t}} = \frac{d_{x+t}}{l_x} = \frac{l_{x+t}-l_{x+t+1}}{l_x}\]

Pero observemos que:
\[\frac{l_{x+t}-l_{x+t+1}}{l_x} = \frac{l_{x+t}}{l_x}- \frac{l_{x+t+1}}{l_x} = {}_tP_x - {}_{t+1}P_x = S_x(t)-S_x(t+1)\]

No definamos que "restar un intervalo" implica dividir, en este caso como tal, es una resta directa de probabilidades que se interpretan sobre una recta.
\\
Se hace mención, ya que \textbf{si lo interpretamos sobre una recta se trata de "restarle" al intervalo más CHICO (\(S_x(t)\)) el otro intervalo más grande (\(S_x(t+1)\))}. Sé que suena un poco extraño, pero es una forma para mí de recordarlo.

\sectiom*{Esperanza de Vida ABREVIADA}
Puntos clave:
\begin{itemize}
    \item Mencionamos que es abreviada porque maneja la variable t (tiempo) como discreta, años completos.
    \item La función piso sobre \(T_x\) que es el tiempo de vida futuro de una persona de edad x, nos da \(K_x\)
    \item \(K_x\)es el tiempo de vida futuro de una persona de edad x EN AÑOS COMPLETOS, se lee de la misma forma que \(T_x\) claro, ya que parte de ella aplicada a una función que solo hace redondearlo, por ende el concepto de \textbf{años completos}.
    \item ¿Cómo calculabas la esperanza en probabilidad? Esto era una suma de los valores x (valores que puede tomar x) multiplicado por su probabilidad, tal que:
    \[E(x) = \sum_{\forall i} x_i \cdot P(x_i)\]
\end{itemize}

\[(e^0_x) = \sum_{t=0}^{w-x} t \cdot P(t) = \sum_{K_x=0}^{w-x} K_x \cdot P(K_x)\]

Entonces si \(x=0\):
\[(e^0_0) = \sum_{t=0}^{w} t \cdot P(t)\]

Recordemos que:
\begin{itemize}
    \item \(t\) y \(K_x\)) son los \(x_i\) valores (en la formula de esperanza de probabilidad) el cuales puede tomar.
\end{itemize}

Entonces, ¿Cómo se calcula \(P(K_x) = P(t)?\)
\\
Veamos que 





\end{document}

