\documentclass{article}
\usepackage{graphicx} % Required for inserting images
\usepackage[a4paper, margin=1in]{geometry}
\usepackage{amssymb}


\title{Álgebra Lineal}
\author{Héctor Astudillo}
\date{December 2024}

\begin{document}

\maketitle

\section{Section 1.1 - Vector Spaces}
\subsection*{Vectors}
Let \( F \) be a field. A vector space over \( F \) (denoted by \( EV/F \)) is a set \( V \) (\textbf{whose} elements are vectors that belonging to the vector space), equipped with two operations that satisfy 8 axioms:

\begin{itemize}
    \item \(EV1\) Commutativity of addition: 
    Let \( u, v \in V \), then \( u + v \in V \).
    \item \(EV2\) Associativity of addition:
    For all \( u, v, w \in V \), we have that \( u + (v + w) = (u + v) + w \).
    \item Commutativity of addition: 
    For all \( u, v \in V \), we have that \( u + v = v + u \).
    \item \(EV3\) Elemento neutro aditivo - Additive identity: 
    Exists \( 0 \in V \) such that \( u + 0 = u \) para todo \( u \in V \).
    \item \(EV4\) Elemento inverso aditivo - Additive inverses:
    For each \( u \in V \), exist a unique element \( -u \in V \) such that \( u + (-u) = 0 \).
    \item \(EV5\) Scalar Product/Multiplication: 
    For all \( c, d \in F \) and \( u \in V \), it satisfies that \( (c \cdot d) \cdot u = c \cdot (d \cdot u) \).
    \item \(EV6\) Neutro/Identidad escalar - Identity element of Scalar Multiplication: 
    For all \( u \in V \), it follows that \( 1 \cdot u = u \), where \( 1 \) is the identity element of the product from the field \( F \).
    \item \(EV7\) Distributivity of scalar over field addition: 
    For all \( c, d \in F \) and \( u \in V \), it holds that \( (c + d) \cdot u = c \cdot u + d \cdot u \).
    \item \(EV8\) Distributivity of scalar over vector addition: 
    For all \( c \in F \) y \( u, v \in V \), se cumple \( c \cdot (u + v) = c \cdot u + c \cdot v \).
\end{itemize}

\(F^n\) means n-tuples (adas) of numbers from the field \(F\)

For example, if \(x \in F^n\), then we are talking about a vector with \(n\) entries, whose coefficients belon tot the field \(F\).
\\
\subsection*{Matrices}
Let \(m,n \in N\) y \(F\) a field, we define \(F^{mxn}\) as the set/collection of all \(m x n\) matrices with entries from \(F\)
\\
\\
Remember that:
\\
The sum of matrices is equal entry by entry.
\\
\\
Let us formally define the addition and scalar product on the vectorial space \(F^{mxn}\).
\\

Addition:

Let \(M, N \in F\)...

For every \(i \leq m\) and every \(j \leq m\) we have that \((M + N)_{ij} = M_{ij} + N_{ij}\)
\\

Scalar Product

\(\alpha \in F\), \(m \in F^{mxn}\) such that \(\alpha M \in F^{mxn}\)

Then, \(i \leq m\) and every \(j \leq m\) we have that \((M + N)_{ij} := \alpha \cdot M_{ij}\)
\\
\\
Now we can see that this kind of spaces satisfy the remaining 6 axioms.
\\
\\
Another example... now, we are going to work with functions.

\subsection*{Functions}
Let \(F\) be a field and \(S\) \textbf{a non-empty set}
\\
We define \(F^S\) as \((f | f: S \rightarrow F)\)
\\
In this case, \(S\) is the codomain and \(F\) (the field) is the domain.
\\

Notation:

We can express a function in 2 different ways...
\begin{itemize}
    \item \(f : [a,b] \rightarrow R\)
    \item \(F \in R^{[a,b]}\)
\end{itemize}

Remember the following from Calculus 1:

Let \(f,g \in R^{[a,b]}\) the addition of this functions \(f+g\) maps from \([a,b]\) to \( \mathbb{R}, \forall x \in [a,b]\)

Then, \((f+g)(x) = f(x) + g(x)\)
\\
\\
So it´s the same for our example, now we are going to define the scalar product:

Let \(\alpha \in F\) and \(f \in F^S\)

We want to get \(\alpha \cdot f \in F^S\)

\(\forall t \in S\)

\((\alpha \cdot f)(t) := \alpha \cdot f(t)\)
\\
\\
We already have the addition and scalar product now we can compute the remaining 6 axioms, but there is something else if we want to prove that a functions is equal to another function, we need to prove that the codomain, domain and rule of correspondence it´s the same for both.
\\

Lets shall the \(EV8\)

Let \(\alpha \in F\) and \(f,g \in F^S\)

PD. \(\alpha \cdot (f+g) = \alpha f +\alpha g\)
\\

Domain:

\(dom(\alpha(f+g)) = S\)

\(dom(\alpga f + \alpha g) = S\)
\\

Codomain:

\(cod(\alpha(f+g)) = F\)

\(cod(\alpga f + \alpha g) = F\)
\\

For the rule of correspondence:

Let \(t \in S\)

\((\alpha(f+g))(t) = \alpha \cdot (f+g)(t)\)

\(=\alpha \cdot (f(t) + g(t))\) the field has distributivity

\(= \alpha \cdot f(t) + \alpha \cdot g(t)\)

\(=(\alpha f)(t) + (\alpha g)(t)\)

\(=(\alpha f + \alpha g)(t)\)

\subsection*{Polynomials}

A polynomial with coeficients in \(F\) is:

\[
\sum_{i=0}^{n} a_i x^i = a_0 + a_1 x^1 + a_2x^2 + a_3x^3... +a_nx^n} 
\] 
where \(n \in \mathbb{N} \cup {0}\) and \(\{a_i : 0 \leq i \leq n\} \subsetneq F\)
\\
\\
\(F[x]\) is the collection of all the polynomials with coefficients in \(F\)
\\
\\
Now we need to define the addition and scalar product to shall the remaining 6 axioms. Let´s see how to define formally a polynomial:

Let \(p \in F[x] \rightarrow \exists n \in \mathbf{N}\cup\{0\}\)


\(\qquad \qquad \qquad \quad \exists \{a_i : 0\leq i \leq n\} \subset F\)

Such that 

\[ \(p = \sum_{i=0}^{n} a_ix^i\). \]

The same if we have \(q \in F[x]\)

Let \(q \in F[x] \rightarrow \exists m \in \mathbf{N}\cup\{0\}\)


\(\qquad \qquad \qquad \quad \exists \{b_i : 0\leq i \leq m\} \subset F\)

Such that 

\[ \(q = \sum_{i=0}^{m} b_ix^i\). \]
\\
\textbf{Notice that each polynomial has its own highest degree term and set of coefficients}
\\
The degree term n and m could be any value in Natural field, anyone, so one could be bigger than other, that is the reason why we define \(k\) 

Let \(k := max\{m,n\} \rightarrow \forall n < i \leq k\) \((a_i :=0)\)

\(\qquad \qquad \qquad \qquad \qquad \forall m < i \leq k\) \((b_i :=0)\)

This help us to complete, force the polynomial to have the same degree term (the highest)
\\

Let´s shall and make an abstract definition of addition and scalar product with polynomials and then, prove the remaining 8 axioms.
\\
\\
Addition
\\
We are going to use \(k\) the highest value between m and n that complete the polynomial.

Let \(p, q \in F[X]\)

We define the addition of polynomials as follow:
\[
\(p+ q = \sum_{i=0}^{k} (a_i + b_i) x^i\) \]
\\
\\
Scalar Product

Let \(\alpha \in F\) and \(p \in F[x]\)

Remember how to define formally an polynomial:

\(p \in F(x) \rightarrow \exists n \in \mathbf{N}\cup \{0\}\)

\(\qquad \qquad \quad \exists \{a_i : 0 \leq i \leq m \} \subset F\)
\\
\\
\textbf{Remember this order:}

\textbf{Exist a degree term, the polynomial has a limit, you have to define this.}

\textbf{Then you have to define the coefficient of this polynomial, even define the close interval in the same line, this is the index that can take the coefficients, \(i\)}.
\\
\\
Well I notice that I said "the remaining 6 axioms on each section", which is incorrect, there are 8 axioms cause we define the closure of addition and closure under scalar multiplication. In total are 10 axioms, not 8.
\\
Now let´s shall the \(EV1\) axiom. Commutativity of addition.

Let \(p, q in F[x]\)

P.D. \(p+q = q+ p\)
\\

Now you need to define its coefficients and degree terms of both polynomials, as well as \(k\) the maximal term degree.
\\

Thus, \(p+q = \sum_{i=0}^{k}(a_i+b_i)x^i\)
\\

\(a,b \in F\) and \(F\) is a field with commutativity of addition...

\(\forall \; 0 \leq i \leq k\) ( \(a_i + b_i = b_i + a_i\) )

Then, \(p+q = \sum_{i=0}^{k} (b_i+a_i)x^i = p+q\)
\\
\\
The following outcomes provide us the fundamental properties of a vector space


\subsection*{Theorem 1. Cancellation law for vector addition}

If \(x, y, z\) are elements of a vector space \(V\) such that \(x+z=y+z \Rightarrow x= y\)
\\

\textbf{This theorem is useful if we want to prove that something is unique, or even to show that 2 variables are equal as we can see in the context of subspaces}
\\
\\
Suppose that \(V\) is a vector space over \(F\) and prove the following...
\begin{itemize}
    \item 1. Exist a \textbf{unique} vector \(w \in V\) such that \(x+w=x\) for each \(x \in V\)

    Additive identity
    \item 2. For each \(x \in V \) exists a unique \(y \in V\) such that \(x+y=0\)

    Additive inverse
\end{itemize}

\textbf{Solution}
\\
1. We need to see that \(w\) is a unique vector...

Let \(w_1, w_2 \in V\) such that \(x+w_1 = x\) and \(x+w_2=x\) for each \(x \in V\)

We apply the theorem 1 in this case... by the cancellation law of addition we obtain that \(w_1 = w_2\)

Thus exists a unique vector \(w \in V \) such that \(x+w=x, \forall x \in V\)

\subsection*{Theorem 1.2. 0 vector properties}
If \(v\) is a EV/F, then:

1. \(\forall v \in V (0\cdot v = 0_v)\)

2. \(\forall \alpha \in F\) and \(\forall v \in V ((-\alpha)\cdot v = \alpha (-v) = -(\alpha \cdot v))\) 

3. \( \forall \alpha \in F (\alpha \cdot 0_v = 0_v)\)

\section*{Section 1.3 - Vector Subspaces}

Sub implies that there is another set that contains the subset, for this case.

Let \(V \) a EV/F

We say that \(W \subsetneq V\) is a subspace

If the operations of V, the space, are the same in W since is a subspace then, we need to verify two operations and the remaining axioms. However, it is sufficient to prove only three axioms.

\subsection*{Theorem 1.3. 3 axioms that prove a vector subspace}

Let \(V\) a EV/F and \(W \subset V\)

We will use \(\leq\) to define a subset... \(W\leq V\)

So, \(W \leq v \) if and only if:

1. \(0_v \in W\)

2. \(\forall x,y \in W ( x+ y \in W)\) - Closure of addition over the subspace

3. \(\forall \alpha \in F\) and \(\forall x \in W (\alpha \cdot x \in W)\) - Closure of multiplication over the subspace
\\
\\
To prove an 'if and only if' statement remember that you must demonstrate both the forward implication and the reverse implication, proving just one is not enough...
\\

For example, a way to prove that \(0_v \in W\) you must demonstrate that \(0_v = 0_w\) using the theorem 1, cancellation law of addition.

To prove the remaining 2 axioms, remember operations as functions...

\((+) : W^2 \rightarrow W \subset V\)
\\
\\
\textbf{Don´t forget or confuse the notation of ordered pairs}

\(W\) x \(W = W^2 = \{ (x,y)\;|\; x, y \in W\}\)
\\
\\
For this case:

\(W\) x \(W \rightarrow W \subset V\) - This indicates that the addition of any element from V (which is also in W, as the additive identity) with an element of W \textbf{results in an element of both spaces}.

This help us for the statement 1 and 2.

\((x,y) \rightarrow x+y\)
\\

\(F\) x \(W \rightarrow W\) - This is for the third statement.

\((\alpha,x) \rightarrow \alpha x\)
\\
\\
The next steps are the remaining 8 axioms that define a vector space, but this can be really easy as we show:

We want to prove the commutativity of addition

Let \(x,y \in W\) 

Since \(W \subseteq V,\;\; x,y \in V\) and V is a VS/F...

We conclude that \(x+y=y+x\)

\subsection*{Transposed Matrices}
\end{document}

