\documentclass{article}
\usepackage{graphicx} % Required for inserting images
\usepackage[a4paper, margin=1in]{geometry}
\usepackage{amssymb}
\usepackage{amsmath} 
%%
\usepackage[margin=1in]{geometry}

%Esto es para las lineas de "Paradoja de la Urna - Probabilidad I"
\usepackage{fancyhdr}
\pagestyle{fancy}
\fancyhead[L]{Integration Methods}
\fancyhead[R]{Calculus 2}
\fancyfoot[C]{\thepage}
%%

\title{Calculus 2}
\author{Héctor Astudillo}
\date{January 2025}

\begin{document}

\maketitle

\section*{Integration by Sustitution or Variable Change}

Let \(P\) and \(f\) be functions related as follows:

\(P'(x) = f(x), \forall x\) in some interval

And, let \(Q(x) = P(g(x)) \Rightarrow\)

\[
Q'(x) = P'(g(x))\cdot g'(x) - \text{Chain rule}  
= f(g(x)) \cdot g(x)
\]

Using the Second Fundamental Theorem of Calculus we know that

\[
\int f(g(x)) \cdot g'(x) \, dx = P(g(x)) + C \tag*{4.1}
\]

And for the function \(Q\):

\[
Q(x) = P(g(x)) = \int  f(g(x)) \cdot g'(x) dx \tag*{4.2}
\]

Now, here is the point where we need to prove with values for \(u\)...

For this case, we select \(u=g(x)\) where its \textbf{derivative} is \("u' - This is incorrect" = \frac{du}{dx} = g'(x)\)

\textbf{Substituting} the value of \(u\) in 4.2...

\[\int f(u)\cdot \frac{du}{dx} dx = P(u) + C\]

\(dx, du\) are not values of real numbers, these are symbols, so it is common write:

\[\frac{du}{dx} dx= g'(x) dx = du\]

Then, \textbf{substituting} in 4.2 we obtain:

\[
\int f(u) du = P(u) + C
\]

\subsection*{Simple Example}

Compute \(\int \frac{sen\sqrt{x+1}}{\sqrt{x+1}} dx\)

Solution:

The first step is define a value for \(u\) and its derivative... \textbf{and the value for dx}

\[ u = \sqrt{x+1}\) \text{ and } du = \frac{1}{2\sqrt{x+1}} dx\]

Notice that:
\(dx = 2\sqrt{x+1} du\)

With this we obtain that:

\[
\int \frac{sen \sqrt{x+1}}{\sqrt{x+1}}
dx\]

\[
=\int \frac{sen(u)}{u} \cdot 2\sqrt{x+1} du 
= \int \frac{sen(u) \cdot 2\sqrt{x+1} du}{\sqrt{x+1}} 
= \int sen(u) du
\]

\subsection*{Example with 2 variable changes}
Compute \(\int \frac{x}{\sqrt{1+x^2+\sqrt{(1+x^2)^3}}} dx\)

Look that in this case we also take as the value of \(u\) the term that is twice repeted.

Solution

\[u = 1+x^2, du = 2x \textbf{dx},  \frac{du}{2} = x \textbf{dx} \text{ and } dx= \frac{du}{2x} = \frac{2x dx}{2x}\]


Hence, we make the substitution

\[
\int \frac{x}{\sqrt{1+x^2+\sqrt{(1+x^2)^3}}} dx = \int \frac{x}{\sqrt{u+\sqrt{u^3}}} \cdot \frac{2x dx}{2x} = \int \frac{x\cdot dx}{\sqrt{u+\sqrt{u^3}}}
\]

Notice that we know the numerator, this is equal to \(\frac{du}{2}\), then only solve it:

\[ \int \frac{\frac{du}{2}}{\sqrt{u+\sqrt{u^3}}} = \int \frac{du}{2 \cdot \sqrt{u+\sqrt{u^3}}} = \frac{1}{2} \int \frac{du}{\sqrt{u + \sqrt{u^3}}}\]

Remember that \(\sqrt{u^3} = \sqrt{u}\cdot \sqrt{u^2} = \sqrt{u}\cdot u\)

\[
= \frac{1}{2} \int \frac{du}{\sqrt{u+u\sqrt{u}}} = \frac{1}{2} \int \frac{du}{\sqrt{u(1+\sqrt{u})}} = \frac{1}{2} \int \frac{du}{\sqrt{u} \cdot \sqrt{1+ \sqrt{u}}}
\]

Then we have to do another variable change... notice that is not necessary have 2 equal values for define the variable change

\(v = 1+\sqrt{u}\) and \(dv = \frac{1}{2\cdot \sqrt{u}} du = \frac{du}{2\sqrt{u}} \) - Be careful with the value of u, it is not x

Then, once more we substituting our integration...

\[= \frac{1}{2} \int \frac{du}{\sqrt{u} \cdot \sqrt{1+ \sqrt{u}}} = \int \frac{du}{2\cdot \sqrt{u} \cdot \sqrt{1+\sqrt{u}}}
\]

We back the constant into the integration because doing this we can change by the value of \(dv\)...

\[\int \frac{dv}{\sqrt{v}} = \int \frac{1}{\sqrt{v}} \cdot dv = \int v^{-\frac{1}{2}} \cdot dv\]

Now we ca solve this integration without problem...

\[\int v^{\frac{1}{2}} = \frac{v^{-\frac{1}{2} + 1}}{-\frac{1}{2}+1} = \frac{v^{\frac{1}{2}}}{\frac{1}{2}} + C = 2\cdotv^{\frac{1}{2}} + C
\]

Then we only need substituting again the values

\[=2\sqrt{1+\sqrt{1+x^2}} + C\]

\subsection*{Example \textbf{of definite Integration}}
Compute \(\int_{0}^{1} \frac{x+1}{(x^2+2x+2)^3} dx\)

Let \(u = x^2+2x+2,\;du=2x+2\;dx = 2(x+1)\;dx\) and \(dx=\frac{du}{2(x+1)}\) 

Notice that in this case the value for \(u\) \textbf{appears only once}, and for \(du \text{ and } dx\) \textbf{you need to simplify as much as possible}.

\[
\int_{0}^{1} \frac{x+1}{(x^2+2x+2)^3} dx = \int_{0}^{1}\frac{x+1}{(x^2+2x+2)^3} \cdot \frac{2(x+1)dx}{2(x+1)}
\]

Be careful with what you cancel from the \textbf{denominator and numerator}, in this case we cancel \(x+1\) \textbf{from both}...

\[
= \int_{0}^{1} \frac{2(x+1)dx}{u^3\cdot 2} = \frac{1}{2} \int_{0}^{1} \frac{(x+1)}{u^3} = \frac{1}{2} \int_{0}^{1} \frac{du}{u^3}
\]

The remaining steps are easier...

\[= \frac{1}{2} \int_{0}^{1} \frac{du}{u^3} = \frac{1}{2} \int_{0}^{1} u^{-3} \cdot du \text{ it is easier to solve this kind of integration =  }\frac{1}{2}\cdot \frac{u^{-2}}{-2} + C
\]

\[
\frac{1}{2} \cdot \frac{\frac{1}{u^2}}{-2} + C = \frac{1}{-4u^2} +C
\]

Then we only need to evaluate it with our values given...

\subsection*{Theorem 1 - The integration of a composite function with values into the domain as functions}

Let \(g\) be a \textbf{continuous} function with continuous derivative on an interval [\(a,b\)] and \(f\) another continuous functions on the interval [\(c,d\)] such that \(g(t) \in [c,d], \forall t \in [a,b]\). Then:

\[
\int_{a}^{b}f(g(t))\cdot g'(t) dt = \int_{g(a)}^{g(b)} f(x)dx
\]

Demonstration:

Let \(F\) a primitive of \(f\), in other words...

\(F'(x) = f(x), \forall x \in [c,d]\)
\\

\textbf{The function \(f \circ g\) is the primitive of \((f\circ g) g'\)} - Notice, this is the first equality that we gave and it is the formula for \textbf{composite} functions. Even is an important statement that we will use into the demonstration. Behold:
\\

\((F \circ g)'(t) = F'(g(t))\cdot g'(t)\)) - We apply only the formula of \textbf{composite} functions.

    \(= f(g(t))\cdot g'(t), \forall t \in [a,b]\)

Then, we will integrate the previous result, and then, we will use the change of variable formula:

For this case \(x=g(x) \text{ and } dx = g'(x)\)

When we make the substitution over the previous result we get the following outcomes:

\[
\int_{a}^{b} f(g(t))\cdot g'(t)dt = \int_{a}^{b} f(u) du \text{, solving it } = F(u) du = F(g(t))\cdot g'(t)
\]

Now, substituting with the values \(t= a,b\) and using the second fundamental theorem of calculus, we obtain:

\[(F\circ g)(b) - (F \circ g) (a) = F(g(b)) - F(g(a))
\]

Look that we were working with \(F\) its primitive, we may re-writing this as an integration such that:

\[
=\int_{g(b)}^{g(b)} f(x) dx
\]

That makes sense, because the solution of this integration is \(F(x)\) and making the substitution with the values of the integral \(g(a), g(b)\) we obtain again the last equality, \(F(g(b)) - F(g(a))\)

\subsection*{Example: Evaluating in terms of \(u\) by changing the integration limits}

Compute \(\int_{0}^{\frac{\pi}{4}} cos2x\sqrt{4-sen2x} dx\)

Solution

Let \(u = 4-sen2x\) such that \(du = -2cos2xdx\) and \(dx=\frac{du}{-2cos2x}\)

Now, we adjust the integration limits \textbf{to match the substitution}:

\[
x=0 \Rightarrow u = 4- sen 0 = 4
\]
\[x=0 \Rightarrow u = 4- sen 2 (\frac{\pi}{4}) = 4 - (\frac{1}{2} \cdot 2) = 3 \]

Substituting these into the integral...

\[
\int_{4}^{3} (cos2x
\sqrt{4-sen2x}) \cdot \frac{du}{-2cos(2x)} \text{ cancel \(cos2x\) from both sides} = \int_{4}^{3} \frac{(
\sqrt{u} \cdot du)}{-2}\]

\[
= -\frac{1}{2} \int_{4}^{3} \sqrt{u} \cdot du = \frac{1}{2} \int_{3}^{4} u^{\frac{1}{2}}\cdot du = \frac{1}{2} \cdot \frac{u^{3/2}}{\frac{3}{2}} = \frac{2u^{\frac{3}{2}}}{3} \cdot -\frac{1}{2} = \frac{u^{\frac{3}{2}}}{3}
\]

\textbf{Finally, we just need to evaluate at }\(a = 4\) and \(b=3\), even notice that we apply an integration property to work with positive constants in the second equity.

\textbf{Simplify further if needed.}

\subsection*{Theorem Example 1}
To be proven:

\[
\int_{a}^{b} f(x)dx = \int_{a+c}^{b+c} f(x-c)
\] - Notice that we are working with composite functions. Don´t get confused with the values \(b+c, a+c\) from the integration, these are even functions, we can see them as functions.

Solution:

For this example, we can define what it is inside of our last function \(f(x-c)\) as the value \(u\), so \(u = x-c,\;du = 1\), but remember, we are working with functions then \(g(x) = x-c,\;g'(x) = 1\)
\\

Then we have to make an equality such that:

\(g(a+c) = a = a+c-c\)

\(g(b+c) = b = b+c-c\)

Since \(x-c = g(x)\) behold:

\[\int_{a+c}^{b+c} f(x-c)dx = \int_{a+c}^{b+c} f(g(x))dx = \int_{a+c}^{b+c} f(g(x)\cdot g'(x) = F\circ g (x) \Big|_{a+c}^{b+c}\]

\[
F\circ g(b+c) - F \circ g(a+c) = \int_{g(a+c)}^{g(b+c)} f(x) dx= \int_{a}^{b)} f(x) dx\;\; \text{ since g(a+c) = a+c-c, we define this functions at the beginning }
\]



\subsection*{Theorem Example 2}

\subsection*{Theorem Example 3}

\section*{Integration by parts}
The formula for integration by parts is derived from the product rule.
\\
\\
Let \(f \text{ and } g\) derivable functions, then the prima of its product is 

\((f(x) \cdot g(x))' = \text{ use the simple formula } f'(x) \cdot g(x) + f(x) \cdot g'(x)\)

Integrating it, we get the following:

\[\int ((f(x) \cdot g(x))' dx = \int f'(x) \cdot g(x) + \int f(x) \cdot g'(x)\] 

But applying an integration property, we know that 
\[\int ((f(x) \cdot g(x))' dx = f(x) \cdot g(x) + C\]

Then we have that the last integration of our penultimate equality is:

\[\int f(x) \cdot g'(x) = f(x) \cdot g(x) - \int f'(x) \cdot g(x) dx +C \]

This equality is called the formula of integration by parts...

For the case of \textbf{defined integrations} we write:

\[\int_{a}^{b} f(x) \cdot g'(x) = f(x) \cdot g(x) \bigg|_{a}^{b}- \int_{a}^{b} f'(x) \cdot g(x) dx +C \]

\[\int_{a}^{b} f(x) \cdot g'(x) = f(b) \cdot g(b) - f(a) \cdot g(a) - \int_{a}^{b} f'(x) \cdot g(x) dx +C \]

And we also may write the formula as follows in terms of \(u, dv:\)

Let \(u=f(x) \text{ and } dv = g'(x) dx \), hence:

\[ \int udv = uv - \int vdu\]

\subsection*{Example 1: Twice applied the integration by parts, notice the importance of a correct choose of \(u, dv\)}

Compute \(\int e^x x^2 dx\)

Solution:

Let \(u = x^2,\; dv=e^x dx\) such that \(du = dx, v = e^x\), where:

\[\int e^x x^2 dx = uv - \int vdu = x^2 e^x - \int e^x 2x dx = x^2e^x - 2 \int x e^x dx\]

The integration obtained is easier than the original but for solve it we have to apply again the integration by parts method (of course in the new integration):

Compute \('int x e^x dx\)

Where \(u =x, dv = e^x dx\) such that \(du = dx, v=e^x\), where:

\[\int x e^x dx = uv - \int vdu = xe^x - \int e^x dx = xe^x - e^x\]

Therefore,
\[\int e^x x^2 dx = x^2 e^x - 2 \int e^x x dx\]
\[= x^2 e^x -2 (x e^x - e^x) + C = x^2 e^x -2xe^x + 2 e^x + C\]
\[e^x (x^2-2x+2) + C\]

If we choose  another value for \(u, dv\) \textbf{we will still arrive at} a result of course, but it might lend to a more complicated integral than the original one.

\subsection{Example 2}

Compute \(\int ln x \;dx \text{ if } x>0\)

Solution:

Let \(u = lnx,\;du = 1/x dx\) and \(dv = dx,\;v=x\)

Applying the formula of integration by parts \(\int udv = uv - \int vdu\). Behold:

\[\int lndx = lnx\cdot x - \int x\cdot \frac{1}{x} = x\cdot lnx-\int dx = x \cdot lnx -x +C\]

\subsection{Example 3 - Sometimes the values for \(u, dv\) are not explicit, you might create them respect to the function gave.}
\end{document}
